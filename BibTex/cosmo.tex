\documentclass{article}
\usepackage{amsmath,amsfonts}
\usepackage{graphicx}
\usepackage[margin=1.0in,letterpaper]{geometry}
\usepackage{float}


\begin{document}

\title{Reference Template}


\author{Leonardo Sattler Cassara }
\date{March 14, 2014}

\maketitle

difwmf


\bibliographystyle{plain}
\bibliography{cosmo_ref}{}

\cite{Birkinshaw} 

This paper gives a generall introduction to the Sunyaev-Zel'dovich
effect, discussing the theory of this effect and presenting many
published results about it, the overall conclusions and important
aspects of them. The author presents, in many steps, relevant equations that
describe the SZ effect, different instruments used to detect it and also
how astronomers use this phenomena to constrain cosmological parameters by
the observation of galaxy clusters.


\cite{Silva}

This article used the SZ effect to find simulated galaxy
clusters and, with the resulting data, measured the SZ effect in terms
of the \emph{compton-y} parameter. It begins with the definition of this
parameter and describes how the simulation is executed 
and at what threshold the measurements were
taken, finally describing what the resulting maps contained in regard
to galactic structures.  

\cite{Chicago}

This website, from the Astronomy Department of the University of
Chicago, gives an illustrated description of physical details behind
the SZ effect, presenting some research data and discussing them.  

\cite{Giodini}

This paper provides severall relations between X-Ray and
SZ effect detection, how they can together provide severall
informations about the observed galaxy cluster and the downside of each
frequency observation, presenting images at both spectra. It also
provideds a review of the recent literature on SZ effect and X-Ray observations.

\cite{Parsons}

In this video about \emph{Inverse Compton Scattering}, Parsons begins with a
review of compton scattering, then goes into the inverse case,
describing a moving electron and a lower energy photon and how the
scattering results in a higher energy photon and a lower energy
electron. Finally, Parsons describes the components of the \emph{compton-y}
parameter, a measurement of the change in energy of CMB photons.  

\cite{LSS}

This is a publication on the SZ effect that inctroduces both types, the
Thermal and the \emph{Non-Thermal} or \emph{Kinematic SZ effect}, presenting the
parameters related to their detection, and the physical informations that
are possible to be extracted from observations of galaxy clusters with
these paremeters. Simulations illustrating these observations are
presented and the results discussed. It also introduces the different
instruments used to detect the SZ effect. 

\cite{Planck}

This is the homepage for the European Space Agency's Plack satellite,
that gives detailed informations about the most recent results of
this mission that, by means of the SZ effect, detect severall galaxy
clusters, being able to create a catalogue of these structures. The
website provides animations explaining the SZ effect, just as many of
the images generated by the data collected by this satellite. 


\cite{wikipedia}

This Wikipedia article descibes the Sunyaev-Zel'dovich Array (SZA) in
California, giving information about its infrastructure, its precise
location, a picture of the instruments (used in the presentation) and
the main goal of this array. 

\end{document}
