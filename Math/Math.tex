\documentclass{article}
\usepackage{amsmath,amsfonts}
\usepackage{graphicx}
\usepackage[margin=1.0in,letterpaper]{geometry}
\usepackage{float}

% are used to denote commments, everything after it on this line is a comment and while not be processed 

\begin{document}

\title{Useful Equations}
\author{Leonardo Sattler Cassara}
\date{March 14, 2014}
\maketitle

\begin{equation}
X= \frac{\cos \delta \sin (\alpha-\alpha_{0})}{\cos \delta_{0} \cos \delta \cos (\alpha-\alpha_{0})+\sin \delta \sin \delta_{0}} ,
\label{eq:2}
\end{equation}


\begin{equation}
Y=-\frac{\sin \delta_{0} \cos \delta \cos (\alpha-\alpha_{0})-\cos \delta_{0} \sin \delta}{\cos \delta_{0} \cos \delta \cos (\alpha-\alpha_{0})+\sin \delta \sin \delta_{0}} ,
\label{eq:3}
\end{equation}


\begin{equation}
x = f\left(\frac{X}{p}\right) + x_{0} ,
\label{eq:4}
\end{equation}


\begin{equation}
x = \frac{f}{p}\left(X\cos \theta-Y\sin \theta \right) + x_{0} ,
\label{eq:6}
\end{equation}

\begin{equation}
y = \frac{f}{p}\left(X\cos \theta-Y\sin \theta \right) + y_{0}.
\label{eq:7}
\end{equation}



\begin{equation} % Nice Exampleeeeeeeeee!!!
\begin{gathered}
az = \arctan{\left(\frac{\vec{r'}_{2,1}}{\vec{r'}_{1,1}}\right)}, \\
alt = \arcsin{(\vec{r'}_{3,1})}.
\end{gathered}
\label{eq:3_f}
\end{equation} 


\begin{equation}
\textbf{x}=\textbf{T}\textbf{X},
\label{eq:ta}
\end{equation}


\begin{equation}
\textbf{T} =
 \begin{pmatrix}
  (f/p)a_{11} & (f/p)a_{12} & x_{0} \\
  (f/p)a_{21} & (f/p)a_{22} & y_{0} \\  
  0 & 0 & 1
 \end{pmatrix}.
\end{equation}



\begin{equation}
\textbf{a}=\textbf{B}\textbf{c} ,
\label{eq:9}
\end{equation}



\begin{equation}
\textbf{a}=
\begin{pmatrix}
x_{1}\\
x_{2}\\
\vdots \\
x_{N}
\end{pmatrix}
,
\textbf{B}=
\begin{pmatrix}
(f/p)X_{1} & (f/p)Y_{1} & 1 \\
(f/p)X_{2} & (f/p)Y_{2} & 1 \\ 
\vdots & \vdots &  \vdots \\
(f/p)X_{N} & (f/p)Y_{N} & 1
\end{pmatrix}
, and \textbf{ c}=
\begin{pmatrix}
a_{11}\\
a_{12} \\    
x_{0}
\end{pmatrix}
\end{equation}




\begin{equation}
\textbf{X}=\textbf{T}^{-1}\textbf{x}   .
\label{eq:ta_inv}
\end{equation}


\begin{equation}
\textbf{r}=\textbf{R}+\textbf{$\rho$s} ,
\label{main}
\end{equation}


\begin{equation}
\rho = k^{2}\left(\frac{1}{R^{3}}-\frac{1}{r^{3}}\right)\frac{\dot{
    \textbf{s}}\cdot(\textbf{R}\times\textbf{s})}{\dot{ \textbf{s}}\cdot(\ddot{\textbf{s}}\times\textbf{s})},
\end{equation}

\begin{equation}
r^{2}=\rho^{2}+R^{2}+2\rho \textbf{R}\cdot \textbf{s} ,
\end{equation}



\begin{equation}
s=
\begin{pmatrix}
\cos \alpha \cos \delta\\  
\sin \alpha \cos \delta\\
\sin \delta \\ 
\end{pmatrix} ,
\end{equation}


\begin{equation}
\begin{split}
\dot{\textbf{s}_{2}}= \frac{ \tau_{3}(\textbf{s}_{2}-\textbf{s}_{1})}{\tau_{1}(\tau_{1}+\tau_{3})} + \frac{\tau_{1}(\textbf{s}_{3}-\textbf{s}_{2})}{\tau_{3}(\tau_{1}+\tau_{3})},\\
\ddot{\textbf{s}_{2}}= \frac{2(\textbf{s}_{3}-\textbf{s}_{2})}{\tau_{3}(\tau_{1}+\tau_{3})} - \frac{2(\textbf{s}_{2}-\textbf{s}_{1})}{\tau_{1}(\tau_{1}+\tau_{3})} .
\end{split}
\end{equation}



\begin{equation}
\begin{split}
\text{T=}
\begin{pmatrix}
0.9685  & 0.0235 & 507.6056 \\
-0.0023 & 1.0123 & 517.7182 \\ 
0 & 0 &  1 \\
\end{pmatrix}
\text{for day 17,}\\ 
\text{and T=}
\begin{pmatrix}
0.9928 & 0.0394 & 504.6182 \\
0.0193 & 0.9959 & 491.0023 \\ 
0 & 0 &  1 \\
\end{pmatrix}
\text{for day 21.}
\end{split}
\end{equation} 


\end{document}
