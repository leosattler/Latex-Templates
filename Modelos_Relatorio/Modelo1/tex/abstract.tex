The \textit{Voyager} and \textit{Galileo} missions provided images and data that revealed an active and
young surface in Europa, with several geological formations such as domes, pits and
evidence for plate tectonics. 
Tidal heating due to 
interaction with Jupiter plays an important 
role in this scenario, and several studies suggest the existence of a liquid subsurface ocean to explain the observed 
features. Proposed models rely on the presence of a floating ice shell, 
where partial melting takes place with activation energy aided by tidal dissipation. Buoyant warm material would reach the surface to form many of the observed features on Europa's surface. However the process that drives this convection is only marginally understood, warranting further investigation. Here we present a 2D numerical model to illustrate 
the possibility of convection in Europa's interior, testing two different scenarios with ice shell depth of $15 \ \textup{km} $ and $20 \ \textup{km}$. With a bottom temperature $T_{\textup{bott}} = 270 \ K$, 
a surface temperature $T_{\textup{surf}} = 100 \ K$ and under Newtonian rheology, we reproduce 
a convective thermal model that is consistent with the presence of a subsurface liquid ocean, and with the terrain's morphology (such as height and extension of pits and ridges). 
\\
\textbf{Keywords:} Europa, Tidal heating, ocean, convection, rheology. 