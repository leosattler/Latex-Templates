As miss�es \textit{Voyager} e \textit{Galileo} forneceram imagens e dados que revelaram uma superf�cie jovem e ativa em Europa, com diversas forma��es
geol�gicas como domos, depress�es e tect�nica de placas. 
O aquecimento por mar�s devido � sua intera��o com J�piter tem um papel 
importante neste cen�rio, e diversos estudos sugerem a exist�ncia de 
um oceano l�quido subsuperficial como explica��o para as forma��es geol�gicas observadas. Os modelos propostos se baseiam na presen�a de uma casca de gelo flutuante, onde derretimento parcial ocorre a partir de uma energia de ativa��o proveniente da dissipa��o das mar�s. Material quente boiaria at� formar v�rias das estruturas observadas na superf�cie de Europa. Entretanto, 
este processo convectivo n�o � muito entendido, necessitando mais investiga��o. Apresentamos aqui um modelo num�rico 2D para ilustrar a possibilidade de convec��o no interior de Europa, testando diferentes cen�rios de profundidade da casca de gelo (de $15 \ \textup{km} $ a $20 \ \textup{km}$). Com uma temperatura de fundo $T_{\textup{fundo}} = 270 \ K$
e uma temperatura de superf�cie $T_{\textup{sup}} = 100 \ K$, sob reologia Newtoniana, n�s reproduzimos um modelo t�rmico convectivo que � consistente com a presen�a de um oceano l�quido sob a superf�cie de Europa, e com as estruturas nela encontradas (como altura e extens�o dos domos e depress�es). 
\\
\textbf{Palavras chave:} Europa, Aquecimento por mar�s, oceano, convec��o, 
reologia.